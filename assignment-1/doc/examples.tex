\section{Examples} \label{sec:examples}
This section gives some examples on the use of different packages.

\subsection{Basics} \label{sec:basics}

\subsubsection{Itemize} \label{sec:itemize}
The following items are displayed with nice\footnote{Really?} squares as bullet points:
\begin{itemize}
	\item One
	\item Two
	\item Three
\end{itemize}

\subsubsection{References} \label{sec:references}
\cref{sec:itemize} and \cref{sec:basics} were part of \cref{sec:summary}. Also, referencing handles appendices like \cref{app:a} or \cref{app:b}.

\subsubsection{Citation} \label{sec:citation}
Citation is an important part of every report \cite{Tietze2019}.


\subsection{Math} \label{sec:math}

\subsubsection{Tensor}

Preceding indices can be created using the the \verb|tensor| package. The documentation can be obtained from \url{https://mirror.easyname.at/ctan/macros/latex/contrib/tensor/tensor.pdf}.

\begin{equation}
	p =
	\begin{pmatrix}
		u\\
		v\\
		1
	\end{pmatrix}
	= K \tensor[^C]{p}{} = K  % note the empty bracket for the post index
	\begin{pmatrix}
		x\\
		y\\
		z\\
		1
	\end{pmatrix}
\end{equation}

\subsubsection{Align \& Case}

The \verb|amsmath| package features the \verb|aligned| and the \verb|align| environment for setting multi-line equations. The main difference
lies in the equation numbering. The \verb|case| statement allows curly brackets for case distinction.

\begin{equation}
	\begin{aligned}
		PSD_{x_{e}, n}(f) &= |G_n(f)|^2 PSD_{n}(f) \\
		&= \begin{cases}
			n|\frac{-CP_a}{1+CP_a}|^2 &\text{for } \SI{1}{\Hz} \leq f \leq \SI{1}{\kHz} \\
			0 & \text{otherwise } 
		\end{cases}
	\end{aligned}
	\label{psd_n}
\end{equation}

\begin{align}
	f_c &= \SI{50}{\Hz}, \\
	k_p = \frac{1}{3 G_{notch}(\num{i} \omega f_c) P(\num{j} \omega f_c)} &= \num{8.05E4}, \\
	T_i = \frac{1}{f_i} = \frac{10}{f_c} &= \SI{0.2}{\s}, \\
	f_d = \frac{f_c}{3} &= \SI{16.67}{\Hz}, \\
	f_t = 3 f_c &= \SI{150}{\Hz}.
\end{align}

\begin{align}
	Z_{11} &= \left.\frac{V_1}{I_1}\right\vert_{I_2=0} = Z_A+Z_C, &
	Z_{12} &= \left.\frac{V_1}{I_2}\right\vert_{I_1=0} = Z_C, \\
	Z_{21} &= \left.\frac{V_2}{I_1}\right\vert_{I_2=0} = Z_C, &
	Z_{22} &= \left.\frac{V_2}{I_2}\right\vert_{I_1=0} = Z_B+Z_C.
	\label{eq2}
\end{align}

\subsubsection{Matrices}

\begin{equation}
	K = \begin{pmatrix}
			f_{x, rgb} & 0 & c_{x, rgb} & 0\\ 
			0 & f_{y, rgb} & c_{y, rgb} & 0\\ 
			0 & 0 & 0 & 0
		\end{pmatrix}.
\end{equation}

\subsubsection{Sums and Integrals}

\begin{equation}
	\eta = \sum_{i = 0}^{n_{cluster}} \frac{1}{{d_i}^2}. 
\end{equation}

\begin{equation}
	\begin{aligned}
	\sigma_{x_e} &= \sqrt{\int_{\SI{1}{\Hz}}^{\SI{1}{\kHz}} \sqrt{PSD_{x_{e}, x_{t}}^2(f) + 
					PSD_{x_{e}, n}^2(f)} \,df}\\
	 &= \sqrt{\int_{\SI{1}{\Hz}}^{\SI{100}{\Hz}} \sqrt{PSD_{x_{e}, x_{t}}^2(f) + 
	 PSD_{x_{e}, n}^2(f)} \,df + \int_{\SI{100}{\Hz}}^{\SI{1}{\kHz}} PSD_{x_{e}, n}(f) \,df}\\
	 &= \SI{8.19E-4}{\m}.
	\end{aligned}
\end{equation}

\subsection{Code}

Simple inline code can be added with the \verb|verb| statement.

\clearpage
\subsection{Figures}

\begin{figure}
	\centering
	\includegraphics[width=14cm]{bode_filters.eps}
	\caption{Bode plots}
	\label{bode}
\end{figure}

Something is shown in \cref{bode}. However, \Cref{sub} is kinda useless.

\begin{figure} [!ht]
	\centering
	\begin{subfigure}[b]{0.45\textwidth}
		\centering
		\includegraphics[width=5cm,trim={0 0 0 15cm},clip]{fig1.png}
		\subcaption{Original Image}
		\label{sub}
	\end{subfigure}
	\hfill
	\begin{subfigure}[b]{0.45\textwidth}
		\centering
		\includegraphics[width=5cm,trim={0 0 0 15cm},clip]{fig1.png}
		\subcaption{Blurred ($\sigma=1$)}
	\end{subfigure}
	\vskip\baselineskip
	\begin{subfigure}[b]{0.45\textwidth}
		\centering
		\includegraphics[width=5cm,trim={0 0 0 15cm},clip]{fig1.png}
		\subcaption{Blurred ($\sigma=2$)}
	\end{subfigure}
	\hfill
	\begin{subfigure}[b]{0.45\textwidth}
		\centering
		\includegraphics[width=5cm,trim={0 0 0 15cm},clip]{fig1.png}
		\subcaption{Blurred ($\sigma=3$)}
	\end{subfigure}
	\caption{A $2 \times 2$ subfigure.}
	\label{subfigure2}
\end{figure}

\clearpage
\subsection{Circuit diagrams}
\begin{figure}
\centering
\begin{circuitikz}
	\draw
	(0,0) node[op amp](amp){}
	(amp.-) |- (-1,2) to[R, l=$R$] (1,2) -| (amp.out)
	(amp.out) to[short] (1.5,0)
	(1.5,0) to[short, -o, l^=$U_{out}$] (2.3,0)
	(-3.5,0.5) to[sC, l=$C_x$] (amp.-)
	(amp.+) -| (-1.2,-1)
	(-1.2,-1) node[rground]{}
	(-3.5,-1) node[rground]{}
	(-3.5,-1) to[sV, l=$U_q$] (-3.5,0.5)
  	;
  	\draw  (amp.-) to[short,*-] ++(0,0);    
    \draw  (amp.out) to[short,*-] ++(0,0);    
\end{circuitikz}
\caption{Random transimpedance amplifier for capacitance sensing}
\label{transimpedance}
\end{figure}

\begin{figure} [t]
\centering
\begin{circuitikz}
	\draw
	(1,0) node[anchor=east]{$U_{TIA}$}
	to[R, o-*, l=$R_1$] (2.75,0)
	(2.75,-2) node[rground]{}
	(4,-0.5) node[op amp](amp1){}
	(2.75,-2) |- (amp1.+)
	(3,1.5) to[sDo, l^=$D_1$] (5,1.5)
	(2.75,0) |- (3,1.5)
	(5,1.5) -| (amp1.out)
	(amp1.out) to[sDo, *-*, l^=$D_2$] (7, -0.5)
	(4,3) to[R, l=$R_2$] (6,3)
	(2.75,1.5) |- (4,3)
	(6,3) -| (7, -0.5)
	(2.75,1.5) to[short, *-] (2.75,1.5)
	(9,0) node[op amp](amp2){}
	(7, -0.5) to[short] (amp2.+)
	(amp2.-) |- (9,1.5)
	(9,1.5) -| (amp2.out)
	(amp2.out) to[twoport, *-o] (13, 0) node[anchor=west]{$U_{out}$}
  	;
  	\draw (4,-0.5) node{\sf $OP_1$};
  	\draw (9,0) node{\sf $OP_2$};
  	\draw (11.625,0) node{\sf $LP$};
\end{circuitikz}
\caption{Another medium badass circuit.}
\label{fig:demod_schem}
\end{figure}

\begin{table}
	\centering
	\begin{tabular}{l | c  c  c}
		\toprule
			Object & Correct & False negative  & False positive \\
		\midrule
			Book & 2 & 0 & 2 \\
			Cookie Box & 9 & 1 & 0 \\
			Cup & 8 & 3 & 4\\
			Ketchup & 8 & 2 & 4 \\
			Sugar & 10 & 0 & 3 \\
			Sweets & 5 & 5 & 1\\
			Tea & 6 & 4 & 1\\	
		\midrule
			Total & 42 & 15 & 15\\
		\bottomrule
	\end{tabular}
	\caption{Detection results}
	\label{tab}
\end{table}